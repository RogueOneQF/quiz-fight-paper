\section{QuizFight's server}
QuizFight relies on a Node.js server and a MongoDB database. We choose these technologies consequently to the choice of the JSON format for exchanging data. In fact, since Node.js uses JavaScript as a language, it is well suited for dealing with JSON messages. The same motivation applies for MongoDB. It is a NoSQL database, where documents can be thought as JSON files. The flexibility of both the communication and the saving cajoled us to use this stack. Further, there are no strong evidences, based on performance and overhead, against the use of a Node.js server rather that a Java one. 

For interacting with the database we use a well-known Node.js's package: \texttt{mongoose.js}. Following the Model-View-Controller design pattern we define both models and controllers server side and let the clients provide the View-part. For exposing our APIs, we use \texttt{Express.js}. Each API is represented by a \texttt{Route} object. No access is directly made from the routes to the model objects. Instead, they use some methods provided by the controllers, exactly as stated by the MVC pattern. 

The server is a fundamental component in QuizFight. In fact, in addition to store users and duels' data it performs the binding $<$player, opponent$>$ and keeps the actual questions. Duels are thus generated server side and sent, on demand, to the clients, round by round.

Users are identified based on their Google Games' username. This provide an effective way to retrieve their information and to bind them in duels. We could have used MongoDB identifiers, but we instead used usernames for their simplicity. In fact, using MongoDB's ids would required too much effort. Clients are unaware of how the server identifies users, so the best we can di client side is to send usernames. Then, on the server, for each operation involving users we should have retrieved user's information. This pattern, although possible, rapidly deteriorates to a lot of unnecessary requests. We then prefer using usernames, which we know by Google's assurance to be unique. 

