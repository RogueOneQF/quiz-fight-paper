\section{Limitations and further work}\label{sec:limitations}

As said, in this very first version a user must sign in with Google Games.
This is necessary because we identify users with GGames usernames.

Anyway, since this is an Android application, users must have a Gmail
address and must have registered that in Android.

Furthermore, if users are interested in Games they will probably have
a GGames account.
This is why we don't think this is a big limitation.

There are many solutions that address this limitation.
For example we could provide a full enrollment mechanism, asking users to sign
up with email and password. We chose not to do that because it would required
an effort out of the scope of this application.  \\

Another feasible line for further work is represented by offline duels.
Currently QuizFight supports only online multi-round duels against
friends on Facebook, but it could be used for mono or multi-round duels
without the need of Internet connectivity.

In this case, a cheaper connection technology, such as Bluetooth or
WiFi-Direct, is suitable for connecting two or even more devices.
This would partially solve the aforementioned mandatory-sign-in limitation,
since such duels are feasible even without a sign up (but of course they would
not be considered in achievements and Google-related statistics). \\

Since that Quiz Fight is under development, in order to be able to fight
a friend, both the players need to be added as tester of the app
(and of course they need to be friends on Facebook).

Currently, QuizFight supports only true/false or multiple-choice questions.
A clever application could offer support for different question types, such as
open questions.
That would require a natural language parser, probably supported by some kind of AI
algorithms.
We could also provide funnier way to dare players, such as group duels.
