\section{Limitations}
As said, in this very first version a user must sign in with Google Games. This is necessary because we identify users with Games' usernames. Anyway, since this an Android application, users must have a Gmail address and must have registered that in Android. Further, if a user was interested in Games they would probably have had a Games account. This is why we don't think this is a big limitation. A number of solutions are possible to address that. For example we could provide a full enrollment mechanism, asking users to sign up with email and password. We chose not to do that because it would required an effort out of the scope of this application. 

Another feasible line for further work is represented by offline duels. Currently QuizFight supports only online multi-round duels, but it could be used for mono or multi-round duels without the need of Internet connectivity. In this case, a cheaper connection technology, such as Bluetooth or WiFi-Direct, is suitable for connecting two or even more devices. This would partially solve the aforementioned mandatory-sign-in limitation, since such a duels are feasible even without a sign up (of course losing achievements and Google-related statistics). 

Currently, QuizFight supports only true/false or multiple-choice questions. A more intelligent application could offer support for different question types, such as open questions. That would require a natural language parser, probably with some kind of AI algorithms behind. We could also provide funnier way to dare players, such as group duels.