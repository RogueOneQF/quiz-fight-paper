\section{Application design}\label{ref:design}

QuizFight information model is depicted in Figure \ref{fig:quizfighter}.
The main entry point is \texttt{SignInActivity}.
Our game, in its very first version, requires the user to sign in with Google
Games. Hence, \texttt{SignInActivity} simply shows a button for the sign in.
The user's dashboard is \texttt{HomeActivity}.
Here the player can see their previous duels, both the completed and the
pending ones.
By tapping on a duel, a complete report of the result is shown.
When the duel has not been completed and when a new round is actually
available (i.e. when both the player and their opponent completed the
previous round), a button asking the user for answering the next questions
is shown.
Since showing every duel would incur in a bad user experience we show
only a limited number. We then provide a way to see the complete lists.

\begin{figure}[t]
	\centering
	\includegraphics[width=0.9\linewidth]{QuizFightER}
	\caption{QuizFight information model}
	\label{fig:quizfighter}
\end{figure}

From \texttt{HomeActivity} a user can also see their current situation in terms
of ranks and achievements.
That information is directly retrieved and shown using Google's services.
In addition, QuizFight offers the possibility to take a look at a user's
statistics, such as the number of duels (or rounds) won versus the number of
duels (or rounds) played and the correct answers.
Finally, there is the possibility to sign in with Facebook, for daring the
user's friends.

By tapping on the fight button it is possible to start a new duel.
Here, the adversary can be chosen in three different modalities:

\begin{itemize}
	\item \textbf{random};
	\item selected from the \textbf{leaderboard};
	\item selected from the \textbf{Facebook's friends list}.
\end{itemize}

After \texttt{DuelActivity} is shown, the user is able to answer the
questions of the round. At every answer a visual feedback is given.
As usual, if the answer was correct it is highlighted in green; otherwise the
wrong answer is highlighted in red and the correct one in green.
At the round's ending a dialog is shown to notify the user about their
actual round's score. If both the player and the opponent have terminated
the current round they are both notified about a new round available (if any),
or about the duel's ending (with the final score). 
