\section{Introduction}

The benefits of a gaming approach for learning are well-known. In fact,
people learn better and more when they play games. We are never tired of
playing.
After all, games mean fun. And fun means that our minds are relaxed and open
for gaining new knowledge.
This is why, over the last few years, scientists have begun to commission
more and more scientific games.
The goal is to involve people in science and let them learn new things.

Furthermore, nowadays there are many game shows that let people test their
trivia knowledge.
They are fun and useful because they allow both competitors and audience to
learn new things, while playing.
We develop QuizFight, a trivia quiz application allowing users to test their
trivia knowledge.
Contrariwise to popular game shows, we cannot reward users with money.
Instead they gain experience. 

The basic flow is simple. We identify a bunch of ``hot'' topics and users can
dare each other with duels in three different topics.
Users can also choose to generate random topics for a duel.
Hence, a duel comprises of three different rounds each of which comprises
of five questions.
Each question has an associate difficulty level determining its score.
Difficulty levels range in [1, 3].

QuizFight's ``programmatic'' goal is, on the other hand, to test how easily an
Android application can interact with different well-known service providers,
such as Google Play Services and Facebook.

We found that using their services is pretty simple, even if the learning curve
is not so linear.

In this paper we explain some QuizFight's features and how they are
implemented.
We also discuss some current limitations and provide likely directions for
future work. 

%TODO paper's organization
