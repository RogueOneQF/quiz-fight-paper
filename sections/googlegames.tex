\section{Google Games}\label{sec:ggames}

Google Play Games (GGames from now on) provides a way to easily manage every
detail of a game. By using it, users may complete achievements and quests,
see ranks and experience a consistent feeling on difference devices.
In the administrator side, GGames allows us to configure everything by using
the Google Play Console.

We use GGames for a number of purposes. First of all, since QuizFight is a
game, we define different achievements with an increasing difficulty level.
Examples are ``\textit{Win 100 duels}'', ``\textit{Answer correctly to 1000
questions}'' or ``\textit{Score 45 points in a duel}''.

As usual, by unlocking achievements the users gain experience.
We also collect information about different events, such as the number of
correct answers, the number of duels, or rounds played and so on.
Using GGames for storing those statistics is useful because it provides a
very easy interface (and so we haven't to reinvent the wheel by re-implementing
everything from scratch).
The most important service we use is \textbf{Saved Games}, as discussed below.

\subsection{Sign In}

In this very first version QuizFight requires a Google Games account for the
sign in. We use that for identifying the users server side and for populating
the UI with custom information (such as their username and image).

Using different identity providers would surely improve the user experience,
and we reserve that as future work.
Anyway, we believe that a full client implementation of a federated identity
management system is out of the QuizFight scope, at least for the first
version.

We provide the sign in functionality by means of the Google APIs.
The most important class is \texttt{GoogleAPIClient} which is the main entry
point for almost every service provided by Google.
It uses a builder design pattern allowing us to specify which functionalities
we need.
To improve user experience we implement the so called ``silent sign in''.
By means of Shared Preferences we store a boolean flag that indicates whether
the user has signed in.
In this way there is no need, for the users, to tap the ``Sign In'' button if
they did not previously signed out. 

The \texttt{GoogleAPIClient} instance must be shared through every Activity
needing Google services.
This is why we rely on the Android's ``idea'' of the singleton design pattern.
We basically extend the \texttt{Application} class and save a reference to the
client. In this way the other Activities may get the reference just retrieving
an instance of the \texttt{Application} object, as follows: \\

\texttt{((QuizFightApplication)getApplication()).getClient()} \\

Where \texttt{QuizFightApplication} is our \texttt{Application} class.
We have to register this custom instance in the manifest file, as shown in
Listing~\ref{lst:application}.

\begin{lstlisting}[language=xml, caption={Application declaration}, label={lst:application}]
<application
	...
	android:name=".QuizFightApplication">
\end{lstlisting}

\subsection{Saved Games}

The most important service used in QuizFight is named \textbf{Saved Games}.
It gives us a convenient way to save our players' game progression to Google
servers. We can then retrieve the saved game data to allow returning players
to continue a game at their last save point from any device.
The Saved Games service makes it possible to synchronize a player's game data
across multiple devices.
For example, if you have a game that runs on Android, you can use the Saved
Games service to allow players to start a game on their Android phone, and
then continue playing on a tablet without losing any of their progress.
Saved Games allows us to limit the network traffic with the server and
provides an easy way to store the information even if the device has no
connectivity.
In fact, if the device is disconnected, Saved Games stores locally the data
to be saved and, at the very first reconnection, it sends that to Google's
servers.

A saved game consists of two parts:

\begin{itemize}
	\item an unstructured binary blob: this data can represent whatever we
choose, and our game is responsible for parsing and writing to it;
	\item structured metadata: additional properties associated with the
binary data that allow GGames services to visually present Saved Games in
the default Saved Games list user interface (UI) in the GGames app.
\end{itemize}

A game can write an arbitrary number of Saved Games for a single player,
subject to user quota, so there is no hard requirement to restrict players
to a single save file.
Every game may save up to 3MB blobs. They are stored in the player's Google
Drive space. Google ensures read/write isolation: only QuizFight is able to
read QuizFight saved data.

Figure~\ref{fig:saved-games-hierarchy} shows how we interact with Saved
Games.

\begin{figure}[h]
	\centering
	\includegraphics[width=0.9\linewidth]{SavedGamesHierarchy}
	\caption{Saved Games Hierarchy}
	\label{fig:saved-games-hierarchy}
\end{figure}

\texttt{SavedGamesActivity} is a base abstract class implementing the
interaction with Saved Games. In order to handle configuration changes
it uses a headless fragment, \texttt{SavedGamesFragment}.
It also stores a boolean flag set to \texttt{true} when a configuration
change occurs. In this way its subclasses are able to recognize that,
behaving accordingly. 

To avoid a heavy loading in the UI thread, we use an
\texttt{AsyncTaskLoader}, \texttt{SavedGamesLoader}.
Following Google's best practices, we handle the possibility of conflicts
in Saved Games by loading the most recent instance. Even if we could save
more than one \texttt{Snapshot}, we prefer storing just one.
This is an opportunistic choice: we have no reason to do otherwise.

We use a bunch of classes implementing \texttt{Serializable} in order to
deserialize the blob provided by Google. For this end, we use
\texttt{SavedGames}.
It is basically a companion class providing static methods for serializing
and deserializing bytes. 
